\documentclass[12pt]{article}
\usepackage{fullpage}
\usepackage{minted}
\usepackage{amssymb}
\usepackage{amsmath}
\usepackage{hyperref}
\usepackage[utf8x]{inputenc}

\title{Array manipulations as functional programming}
\author{Jim Pivarski}
\date{\today}

\begin{document}
\maketitle

\setlength{\parskip}{0.5\baselineskip}

\section*{Introduction}

The central features of an array library like Numpy or Awkward Array simplify if we think of arrays as functions and these features as function composition. A one-dimensional array of \mintinline{python}{dtype} $d$ (e.g.\ \mintinline{python}{int32} or \mintinline{python}{float64}) can be thought of as a function from integer indexes to members of $d$. Thus,
\[ \mintinline{python}{array[i]} \]
\noindent becomes
\[ \mintinline{python}{array}: \mathbb{Z} \to d \]
\noindent because given an integer \mintinline{python}{i} $\in \mathbb{Z}$, it returns a value in $d$.  In Python, this function is the implementation of the array's \mintinline{python}{__getitem__} method.

Specified this way, this is a partial function\footnote{\url{https://en.wikipedia.org/wiki/Partial_function}}---for some integers, it raises an exception rather than returning a value in $d$. (Integers greater than or equal to the array's length or less than its negated length, if the array implements Python's negative indexing, are outside the bounds of the array and do not return a value.) It can be made into a total function by restricting the domain to $[0, n)$ where $n$ is the length of the array:
\[ \mintinline{python}{array}: [0, n) \to d. \]

We can choose $[0, n)$ as the domain and work with total functions or $\mathbb{Z}$ as the domain and work with partial functions---it is a matter of the granularity of the type system. Numpy has a single type \mintinline{python}{ndarray} for all arrays, Numba has an array type that depends on the array's dimension, and C++ has a \mintinline{c++}{std::array<dtype, n>} type that depends on the exact size (\mintinline{c++}{n}) of the array, like our functional description above. As we'll see later, a consequence of this specificity is that the return value of some functions will depend on the values given to that function, a feature known as dependent types\footnote{\url{https://en.wikipedia.org/wiki/Dependent_type}}.

In this note, we'll describe arrays as total functions in a dependent type system.

\section*{Multidimensional arrays}

Numpy arrays can have arbitrarily many dimensions, referred to as the array's \mintinline{python}{shape}. The \mintinline{python}{shape} is a tuple of positive integers specifying the length of each dimension: $(n_1, n_2, \ldots, n_k)$ is a rank-$k$ tensor ($k = 1$ is a vector, $k = 2$ is a matrix, etc.).

To get values of $d$ from a rank-$k$ array of \mintinline{python}{dtype} $d$, we must specify $k$ integers, each in a restricted domain $[0, n_i)$. In Numpy syntax, this is an implicit Python \mintinline{python}{tuple} between the square brackets:
\[ \mintinline{python}{array[i1, i2, ..., ik]} \]
\noindent In mathematical syntax, we can represent a $k$-tuple as a cartesian product,
\[ [0, n_1) \times [0, n_2) \times \ldots \times [0, n_k) \]
\noindent so the function corresponding to this array is
\[ \mintinline{python}{array}: [0, n_1) \times [0, n_2) \times \ldots \times [0, n_k) \to d. \]

A function with multiple arguments can be replaced with functions of one argument that each return a function, a process known as currying\footnote{\url{https://en.wikipedia.org/wiki/Currying}}. For example, the function above can be replaced with
\[ \mintinline{python}{array}: [0, n_1) \to [0, n_2) \to \ldots \to [0, n_k) \to d \]
\noindent by noting that
\[ \mintinline{python}{array[i1]} \]
\noindent returns an array of rank $k - 1$ and \mintinline{python}{dtype} $d$, which is a function
\[ \mintinline{python}{array[i1]}: [0, n_2) \to \ldots \to [0, n_k) \to d. \]
\noindent In fact, Numpy's indexing syntax illustrates this clearly:
\[ \mintinline{python}{array[i1, i2, i3] == array[i1][i2][i3]} \]
\noindent for any \mintinline{python}{i1}, \mintinline{python}{i2}, \mintinline{python}{i3} that satisfy a three-dimensional array's domain.

\section*{Record arrays}

Numpy also has record arrays\footnote{\url{https://docs.scipy.org/doc/numpy/user/basics.rec.html}} for arrays of record-like structures (e.g.\ \mintinline{c++}{struct} in C). In Numpy, the named fields and their types are considered part of the array's \mintinline{python}{dtype}, but they are accessed through the same square bracket syntax as elements of the array's \mintinline{python}{shape}:
\[ \mintinline{python}{array[i1, i2, i3][fieldname]} \]
\noindent where \mintinline{python}{fieldname} is a string, the name of one of the record's fields. (Numpy does not allow the \mintinline{python}{fieldname} to be uncurried---to be in the same tuple with \mintinline{python}{i1}, \mintinline{python}{i2}, \mintinline{python}{i3}.)

Since record fields are accessed through a similar syntax, let's consider it part of the array's functional type, making no distinction between \mintinline{python}{shape} elements and field names. For a record type in which string-valued field names $s_1$, $s_2$, \ldots, $s_m$ map to \mintinline{python}{dtypes} $d_1$, $d_2$, \ldots, $d_m$, we can write
\begin{align*}
\mintinline{python}{recarray}: [0, n) \to & s_1 \to d_1 \\
 & s_2 \to d_2 \\
 & \ldots \\
 & s_m \to d_m
\end{align*}
\noindent to represent a one-dimensional record array of length $n$. This is a dependent type because the choice of field name determines the return type of the function.

A multidimensional record array could be described as
\begin{align*}
\mintinline{python}{recarray}: [0, n_1) \to [0, n_2) \to \ldots \to [0, n_k) \to & s_1 \to d_1 \\
 & s_2 \to d_2 \\
 & \ldots \\
 & s_m \to d_m
\end{align*}
\noindent or as
\begin{align*}
\mintinline{python}{recarray}: [0, n_1) \to & s_1 \to [0, n_2) \to \ldots \to [0, n_k) \to d_1 \\
 & s_2 \to [0, n_2) \to \ldots \to [0, n_k) \to d_2 \\
 & \ldots \\
 & s_m \to [0, n_2) \to \ldots \to [0, n_k) \to d_m
\end{align*}
\noindent or any other placement of the field name index within the ordered sequence of dimensional indexes. In general, the string indexes (field names) commute with the integer indexes (dimensions). This is evident in Numpy's syntax:
\begin{align*}
\mintinline{python}{recarray[i1][i2][i3][fieldname]} & \mintinline{python}{ == recarray[i1][i2][fieldname][i3]} \\
 & \mintinline{python}{ == recarray[i1][fieldname][i2][i3]} \\
 & \mintinline{python}{ == recarray[fieldname][i1][i2][i3]}
\end{align*}
\noindent It is also evident if the array is arranged as a rectilinear table, in which $[0, n_1)$, $[0, n_2)$, \ldots $[0, n_k)$ form a $k$-dimensional lattice of bounded integers and the field names are an additional dimension, indexed by a finite set of strings with no predefined order. This dimension of category labels is usually called the ``columns'' and all other dimensions are called ``rows.'' In this picture, rearranging the order of the string index and the integer indexes corresponds to selecting a column before a row, rather than a row before a column.

\section*{Vectorized functions}

Numpy uses so-called ``vectorized'' functions or ``universal'' functions (``ufuncs'') for most calculations\footnote{\url{https://docs.scipy.org/doc/numpy/reference/ufuncs.html}}. (These are not to be confused with vectorized instructions in CPU hardware, but are based on a similar idea.) Any function $f$ that maps scalar \mintinline{python}{dtypes} $d^A$ to $d^B$,
\[ f: d^A \to d^B\mbox{,} \]
\noindent can be lifted to a vectorized function that maps arrays of \mintinline{python}{dtype} $d^A$ to arrays of \mintinline{python}{dtype} $d^B$:
\[ \mbox{ufunc}(f): \left([0, n) \to d^A\right) \to \left([0, n) \to d^B\right)\mbox{.} \]
\noindent Note that the \mintinline{python}{shape} of the array, $[0, n)$ in this case, is the same for the argument type of $\mbox{ufunc}(f)$ as for its return type.

This $\mbox{ufunc}$ functor is a partial application of what would be called $\mbox{map}$ in most functional languages\footnote{\url{https://en.wikipedia.org/wiki/Map_(higher-order_function)}}. The $\mbox{map}$ functor takes a function and a collection, returning a collection of the same length with the function applied to each element. The $\mbox{ufunc}$ functor only takes a function, and its result is applied to collections (arrays) later.

Since arrays are themselves functions, applying $\mbox{ufunc}(f)$ to an array is a composition\footnote{\url{https://en.wikipedia.org/wiki/Function_composition}} of the array with $f$. Thus, the following is true for any \mintinline{python}{i} $\in [0, n)$:
\begin{align*}
\underbrace{\mbox{ufunc}(f)(\mintinline{python}{array})}_{\mbox{\scriptsize array}}(\mintinline{python}{i}) = & \,f(\underbrace{\mintinline{python}{array}(\mintinline{python}{i})}_{\mbox{\scriptsize scalar}}) \\[0.5\baselineskip]
\underbrace{\mintinline{python}{numpy.vectorize(f)(array)}}_{\mbox{\scriptsize array}}\mintinline{python}{[i]} = & \,\mintinline{python}{f(}\underbrace{\mintinline{python}{array[i]}}_{\mbox{\scriptsize scalar}}\mintinline{python}{)}\mbox{.}
\end{align*}
\noindent by associativity of function composition. This composition always applies $f$ to the {\it output} of the array, never the {\it input} (function composition is not commutative).
\begin{align*}
f: & \mbox{ } d^A \to d^B \\
\mintinline{python}{array}: & \mbox{ } [0, n) \to d^A \\
\mbox{ufunc}(f)(\mintinline{python}{array}) = f \circ \mintinline{python}{array} : & \mbox{ } [0, n) \to d^B\mbox{.}
\end{align*}

Using associativity again, we should be able to compose a sequence of scalar functions $f: d^A \to d^B$, $g: d^B \to d^C$, \ldots, $h: d^Y \to d^Z$ before applying them to the array. If the scalar function can be extracted from a ufunc object, it would be possible to compose
\[ \mbox{ufunc}(f) \circ \mbox{ufunc}(g) \circ \ldots \circ \mbox{ufunc}(h) \]
\noindent into a single
\[ \mbox{ufunc}(f \circ g \circ \ldots \circ h): \left([0, n) \to d^A\right) \to \left([0, n) \to d^Z\right) \]
\noindent that can be applied to an array. This is an optimization known as loop fusion\footnote{\url{https://en.wikipedia.org/wiki/Loop_fission_and_fusion}}, and is often faster than making multiple passes over arrays and possibly allocating large temporary arrays between ufuncs. There have been proposals\footnote{\url{https://numpy.org/doc/1.14/neps/deferred-ufunc-evaluation.html}} and external libraries\footnote{\url{https://www.weld.rs/weldnumpy}} to add this feature transparently to Numpy. In principle, it could even be an explicit feature of the ufunc object, but to my knowledge, it has never been implemented as such.

\section*{Array slicing}

Whereas ufuncs compose scalar functions to the output of an array, slicing composes \mintinline{python}{index} arrays (which are functions) to the input of an array.

The fact that array slicing is itself composition may not be obvious because of the way that slicing is presented:
\[ \mintinline{python}{array[i:j]} \]
\noindent does not seem to be a composition of two arrays. The first point to make is that all of Numpy's slicing mechanisms---range slices (Python's \mintinline{python}{slice} operator or \mintinline{python}{start:stop:step}), \mintinline{python}{numpy.compress} with boolean arrays, and \mintinline{python}{numpy.take} with integer arrays---can be rewritten in terms of \mintinline{python}{numpy.take} with integer arrays:
\begin{itemize}
\item A range slice \mintinline{python}{start:stop:step} can be replaced with an integer sequence
\[ \mintinline{python}{range(start, stop, step)} \]
(ignoring the effect of negative \mintinline{python}{start} and \mintinline{python}{stop}).
\item A boolean array \mintinline{python}{mask} can be replaced with \mintinline{python}{numpy.nonzero(mask)}.
\item An integer array \mintinline{python}{index} is already an integer array.
\end{itemize}

As an array, \mintinline{python}{index} is a function $\mintinline{python}{index}: [0, x) \to [0, n)$ that can be composed with $\mintinline{python}{array}: [0, n) \to d$ to produce
\[ \mintinline{python}{array} \circ \mintinline{python}{index}: [0, x) \to d \]
\noindent Note that \mintinline{python}{index} is to the right (transforms the {\it input}) of \mintinline{python}{array}, whereas $\mbox{ufunc}(f)$ put $f$ to the left (transforms the {\it output}) of \mintinline{python}{array}.

Numpy uses the same syntax for this function composition, \mintinline{python}{array[index]}, as it does for function evaluation, \mintinline{python}{array[i]}, which is potentially confusing. Let's illustrate this with an extended example.

\subsection*{Example}

Consider two functions that are defined on all non-negative integers (at least).
\begin{minted}{python}
def f(x):
    return x**2 - 5*x + 10
def g(y):
    return max(0, 2*y - 10) + 3
\end{minted}

They may be transformed into arrays by sampling \mintinline{python}{f}, \mintinline{python}{g}, and $\mintinline{python}{f} \circ \mintinline{python}{g}$ at enough points to avoid edge effects from their finite domains. For \mintinline{python}{f} and \mintinline{python}{g} above, 100 points in \mintinline{python}{g} is enough to accept the entire range of \mintinline{python}{f} when \mintinline{python}{f} is sampled at 10 points.
\begin{minted}{python}
F   = numpy.array([f(i) for i in range(10)])     # F is f at 10 elements
G   = numpy.array([g(i) for i in range(100)])    # G is g at 100 elements
GoF = numpy.array([g(f(i)) for i in range(10)])  # GoF is g∘f at 10 elements
\end{minted}
\noindent Now $\mintinline{python}{F}: [0, 10) \to [4, 46)$, $\mintinline{python}{G}: [0, 100) \to [3, 191)$, and $\mintinline{python}{GoF}: [0, 10) \to [3, 85)$.

Indexing \mintinline{python}{G} by \mintinline{python}{F} can be expressed with square-bracket syntax or \mintinline{python}{numpy.take}, and it returns the same result as the sampled composition \mintinline{python}{GoF}.
\begin{minted}{python}
G[F]               # → [13,  5,  3,  3,  5, 13, 25, 41, 61, 85]
G.take(F)          # → [13,  5,  3,  3,  5, 13, 25, 41, 61, 85]
numpy.take(G, F)   # → [13,  5,  3,  3,  5, 13, 25, 41, 61, 85]

GoF                # → [13,  5,  3,  3,  5, 13, 25, 41, 61, 85]
\end{minted}
\noindent In \mintinline{python}{GoF}, the functions are composed before being transformed into arrays, and in \mintinline{python}{G[F]}, the arrays themselves are composed via integer-array indexing.

Function composition is associative, so we should be able to change the order of two integer-array indexings. To demonstrate this, introduce another array, which need not have integer \mintinline{python}{dtype}.
\begin{minted}{python}
H = numpy.arange(1000)*1.1
\end{minted}
\noindent When we compute \mintinline{python}{H} indexed by \mintinline{python}{G} indexed by \mintinline{python}{F}, it shouldn't matter whether the \mintinline{python}{H[G]} index is computed first or the \mintinline{python}{G[F]} index is computed first, and we see that this is the case.
\begin{minted}{python}
H[G][F]            # → [14.3  5.5  3.3  3.3  5.5 14.3 27.5 45.1 67.1 93.5]
H[G[F]]            # → [14.3  5.5  3.3  3.3  5.5 14.3 27.5 45.1 67.1 93.5]
\end{minted}

\subsection*{Multidimensional slicing}

If Numpy's integer-array indexing for multiple dimensions worked the same was as its range-slicing does, then the above would be trivially extensible to any number of dimensions. However, Numpy's integer-array indexing (called ``advanced indexing'')\footnote{\url{https://docs.scipy.org/doc/numpy/reference/arrays.indexing.html}} couples iteration over integer arrays supplied to each of the $k$ slots in a rank-$k$ array.

To work-around this caveat, consider rank-$k$ integer arrays in each of the $k$ slots, in which the integer array in slot $i$ has shape $(1, \ldots, n_i, \ldots, 1)$. For example, a three-dimensional slice
\[ \mintinline{python}{array[start1:stop1, start2:stop2, start3:stop3]} \]
\noindent can be simulated with integer arrays
\begin{align*}
\mintinline{python}{array[} & \mintinline{python}{numpy.arange(start1, stop1).reshape(-1, 1, 1),} \\
                            & \mintinline{python}{numpy.arange(start2, stop2).reshape(1, -1, 1),} \\
                            & \mintinline{python}{numpy.arange(start3, stop3).reshape(1, 1, -1)]}
\end{align*}
\noindent because Numpy broadcasts the three integer arrays into a common three-dimensional shape, and the symmetry of these arrays decouples their effects in each dimension.

\section*{Awkward Array extensions}









\end{document}
